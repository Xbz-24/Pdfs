\documentclass[12pt]{article}

\usepackage[utf8]{inputenc}
\usepackage{amsmath,amssymb,bbm}
\usepackage{tikz-cd}
\usepackage[margin=1in]{geometry}
\usepackage{lmodern}
\usepackage{microtype}
\usepackage{fancyhdr}
\usepackage{graphicx}
\usepackage[onehalfspacing]{setspace}
\usepackage{titlesec}
\usepackage{hyperref}

\pagestyle{fancy}
\fancyhf{}
\fancyhead[R]{\thepage}
\fancyhead[L]{}
\title{\includegraphics[scale=0.1]{logo.png}\\[0.5cm]\textbf{\Huge Final Exam}\\ \Large Linear Algebra\\ \Large February 19, 2021}
\author{\Large Departamento de Matemáticas, FIUBA}
\date{}

\titleformat{\section}{\large\bfseries}{}{0pt}{}
\titlespacing*{\section}{0pt}{\baselineskip}{0.5\baselineskip}

\begin{document}
	
	\maketitle
	\thispagestyle{empty}
	
	\section{Instructions}\label{sec:instructions}\sectionmark{Instructions}
	\begin{enumerate}
		\item All solutions must be presented with rigor and clarity. Justify your answers thoroughly.
		\item When referencing any results from the course, provide proper citations or references.
		\item Maintain a concise and legible writing style throughout your answers.
		\item Best of luck!
	\end{enumerate}
	
	\vspace{-0.5cm}
	
	\section{Question 1}\label{sec:q1}\sectionmark{Question 1}
	Consider three complex vector spaces \(U\), \(V\), and \(W\), each of finite dimension, and equipped with inner products. Let us examine two linear functions:
	
	\[
	\begin{tikzcd}
		U \arrow[r, "f"] & V \arrow[r, "g"] & W
	\end{tikzcd}
	\]
	
	Prove that if the image of \(f\) (\(\text{Im}(f)\)) is equal to the null space of \(g\) (\(\text{Nu}(g)\)), then the linear function \(h = f \circ f^* + g^* \circ g: V \rightarrow V\) represents an isomorphism.
	
	\textit{Hint:} Consider \(v\), an element in the kernel of \(h\). What can you deduce about \(\langle g(v), g(v) \rangle\)?
	
	\vspace{1cm}
	
	\section{Question 2}\label{sec:q2}\sectionmark{Question 2}
	Let \(V\) be a finite-dimensional vector space over a field \(K\), with \(n\) as its dimension. Additionally, let \(f\) and \(g: V \rightarrow V\) be two endomorphisms of \(V\) such that \(f \circ g = g \circ f\).
	
	\begin{enumerate}
		\item Prove that if \(f\) is diagonalizable, and its characteristic polynomial has simple roots, then \(g\) is also diagonalizable.
		\item Show that if \(f^n = 0\) and \(f^{n-1} \neq 0\), then there exists a polynomial \(P \in K[X]\) such that \(g = P(f)\).
	\end{enumerate}
	
	\vspace{1cm}
	
	\section{Question 3}\label{sec:q3}\sectionmark{Question 3}
	Let \(V\) be a complex vector space of finite dimension equipped with an inner product, and let \(f: V \rightarrow V\) be a linear function. If \(f^* + f = 0\), then prove that \(f\) is diagonalizable, and its eigenvalues have null real parts.
	
	\vspace{1cm}
	
\section{Question 4}\label{sec:q4}\sectionmark{Question 4}
Let \(n \in \mathbb{N}\) and let \(\mathbbm{k}\) be a field.

\begin{enumerate}
	\item[(a)] Let \(A\) and \(B\) be two matrices of \(M_n(\mathbbm{k})\). Prove that there exists an invertible matrix \(C \in M_n(\mathbbm{k})\) such that \(A = CB\) \(\iff \{ x \in \mathbbm{k}^n : Ax = 0 \} = \{ x \in \mathbbm{k}^n : Bx = 0 \}.\)
	\item[(b)] If a matrix \(A \in M_n(\mathbbm{k})\) has the same rank as its square \(A^2\), then there exists an invertible matrix \(D\) such that \(A^2 = DA\).
\end{enumerate}


		\vspace{4cm}
		\section{Question 5}\label{sec:q5}\sectionmark{Question 5}
		Let \(n\) be in \(\mathbb{N}\) and let \(V = \mathbb{R}[X]_{\leq n}\) be the vector space of polynomials with real coefficients that are either zero or have degree at most \(n\).
		
		\begin{enumerate}
			\item[(a)] For each \(t \in \mathbb{R}\), we consider the linear function \(\phi_t: V \rightarrow \mathbb{R}\) defined as follows:
			
			\[\phi_t(p) = p(t) \text{ for every } p \in V.\]
			
			Show that if \(t_0, \ldots, t_n\) are \(n+1\) \textit{pairwise distinct} elements of \(\mathbb{R}\), then the set \(\{\phi_{t_0}, \ldots, \phi_{t_n}\}\) forms a basis for the dual space \(V^*\).
			
			\item[(b)] Now, suppose \(n = 2\), and let \(I: V \rightarrow \mathbb{R}\) be the linear function defined as \(I(p) = \int_{0}^{1}p(x)dx\) for every polynomial \(p \in V\). Express \(I\) as a linear combination of \(\phi_0\), \(\phi_{1/2}\), and \(\phi_1\).
			
			\item[(c)] Continuing with the assumption that \(n = 2\), is it possible to express the function \(I\) from the previous part as a linear combination of two functions \(\phi_{t_1}\) and \(\phi_{t_2}\), by appropriately choosing the numbers \(t_1\) and \(t_2\)?
		\end{enumerate}
		
\end{document}
